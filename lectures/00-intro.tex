%!TEX root = ../main.tex
\section{Ouverture: what is formal category theory} The language of category
theory is built upon a certain number of fundamental notions: among these we
find the universal characterization of co/limits, the definition of adjunction,
(pointwise) Kan extension, and the theory of monads.

It is possible to `axiomatize' these definitions, pretending that they
refer to the 1- and 2-cells of a generic 2-category other than $\Cat$. This
conceptualization is one of the pillars upon which category theory is done: in
some sense, category theory arises when the way in which abstract patterns
interact becomes itself an object of study, and when it is generalized to
several different contexts. In a few words, the aim
of \emph{formal category theory} is to provide a framework in which this process
can be outlined mathematically. Quoting the introduction of \cite{Gray}, that is 
one of the pillars on which the subject is founded,
\begin{quotation} The purpose of category theory is to try to describe certain
general aspects of the structure of mathematics. Since category theory is also
part of mathematics, this categorical type of description should apply to it as
well as to other parts of mathematics. [\dots\unkern]
\end{quotation} The basic idea is that the category of small categories, $\Cat$,
is a 2-category with properties in the same way $\Set$ is a category with properties. 
The aim of formal category theory is to outline these properties, and the assumptions
needed to ensure that a certain 2-category behaves like $\Cat$ in some or some other respects.

Unfortunately, being too na\"ive when performing this process doesn't always
give the `right' answer (by which we mean that it doesn't always build an object
with the right universal property), or at least it doesn't give the right answer in the same
straightforward way in which some categories of algebraic structures can be defined starting
from the category of $\Sets$.

This is ultimately due to the fact that, when moving to the setting of
$\clV$-enriched categories (which is the adjacent step of abstraction from
$\Cat =\Set$-$\Cat$) the theory `behaves differently' in various ways, and some of these
differences prevent $\clV$-categories to be as expressive as one would have
liked it to be (a paradigmatic example of this minor expressivity is the lack of
a \emph{Grothendieck construction} for generic $\clV$-presheaves: seeing how the
Grothendieck construction, a certain rule to relate `presheaves on $B$' and
`fibrations over $B$', ultimately pertains to formal category theory has been
one of the purposes of the early literature on the subject, see
\cite{StreetFibreYoneda1974,street1978yoneda,street1980fibrations}).

The major problem is that the 2-category $\clV\text{-}\Cat$ often doesn't give
enough information about the $\clV$-valued hom-functors in a 2-category. Formal
category theory can be thought as a way to encode the same amount of
information in various other ways: even though it is always possible to do some
constructions by mimicking definitions from $\Cat$ (adjunctions and
adjoint equivalences, extensions by universal 2-cells, \etc), things get a
little hairy when we want to provide the theory with an analogue of a very useful
and basic result as the Yoneda lemma.

In the 2-category $\Cat$, we can use a lax limit construction to ``revert''
set-valued functors on an object $B$ into arrows ``over'' $B$ (we basically glue
together a bunch of fibers $\coprod_b E_b$ projecting onto $B$, in the same
manner we build the étale space of a presheaf $F : B^\opp \to \Set$); in the
2-category $\Cat$ the comma object of $b : 1 \to B$ to $1_B : B \to B$ together
with its projection $b/B \to B$ is a good stand in for the covariant functor
represented by $B$ (more generally, \emph{discrete left fibrations} over $B$
stand in for general functors $B \to \Set$).

In the 2-category $\clV\text{-}\Cat$, we care about $\clV$-valued
$\clV$-functors and we would like to do the same construction there. But for an
object $b$ in a $\clV$-enriched category $B$, the comma $b/B$ is more naturally
an \emph{internal} category (whose object of objects is $\coprod_{x \in B}
B(b,x)$) rather than an \emph{enriched} one (whose objects are morphisms $b \to
x$ in the underlying category of $B$). We have to ensure that the codomain
projection $\clV$-functor $b/B \to B$ from the enriched version of the comma has
a fibration-like properties, and this leaves us with the fundamental problem of
formal category theory: \emph{which additional structure on a 2-category $\clK$ allows to
recognize arrows of $\clK$ playing the same r\^ole of discrete fibrations in
$\Cat$, thus providing with a meaningful notion of Yoneda lemma internal to
$\clK$?} 

It has been in the middle age of southern-emisphere category theory that a
certain number of ways to describe such extra structures have been invented: the
aim of this first chapters is to give a brief account about three (not unrelated)
such attempts. At the moment of writing we count
\subsection{Street and Walters' ``Yoneda structures''} The definition of Yoneda
structure given by Street revolves around the possibility to give a formal
counterpart of the Yoneda embedding $\yon \colon A \to \P A$ with its universal
property. This axiomatization is based on the centrality of the Yoneda lemma
`internal' to a 2-category $\clK$, that has been defined in a fairly
heavy-handed way in Street's previous \cite{StreetFibreYoneda1974}. One of the
main achievements of the subsequent \cite{street1978yoneda} is to obtain an
elegant and concise axiomatization stemming almost completely from universal
properties.
\subsection{Street's ``fibrational cosmoi''} A particular case of Yoneda
structures, where you ask the pseudo-functor $\P : \clK^\coop \to \clK$
characterizing a Yoneda structure to be a right 2-adjoint.
\subsection{Wood's ``proarrow equipments''} A different and more powerful
perspective, where you embed $\clK$ into a second 2-category $\clK^\star$ with a
2-functor $(\firstblank)_* : \clK \to \clK^\star$, which is the identity on
objects and mimicks the behaviour of the embedding $\Cat \to \cate{Prof}$, asking
that
\begin{itemize}
 	\item the 2-functor $(\firstblank)_*$ is locally fully faithful;
 	\item for each 1-cell $f : A \to B$ in $\clK$, every $f_*$ admits a right
adjoint in $\clK^\star$.
 \end{itemize} (See \cite{benabou2000distributors,cofriend} for a thorough
account of the theory of profunctors; since the request that $(\firstblank)_*$
is the identity on objects is a bit of an evil one, the paper
\cite{wood1985proarrows} gives the 2-functors satisfying only the other two
axioms the name of \emph{pro-equipments}).